\documentclass{article}
\usepackage[utf8]{inputenc}
\usepackage[ngerman]{babel}
\usepackage[T1]{fontenc}

\title{\huge CS 102 \LaTeX Übung}
\author{Reto Schwaiger}
\date{\today}

\begin{document}

\maketitle

\section{Wie zum Beispiel: ICH WAR HIER - Reto Schwaiger. Cheers}
\end

\section{Das ist der erste Abschnitt}
\begin{description}
\item Hier könnte auch ein anderer Text stehen - 
\end{description}
\section{Tabelle}
\begin{description}
\item Unsere wichtigsten Daten finden Sie in Tabelle 1.
\end{description}

\begin{table}
\centering
\begin{tabular}{r|c|c|c}
Nothing & Punkte erhalten & Punkte \\\hline
Aufgabe1 & 2 & 4 \\
Aufgabe2 & 3 & 3 \\
Aufgabe3 & 3 & 3
\end{tabular}
\caption{\label{tab:Werte}Diese Tabelle kann auch andere Wert beinhalten.}
\end{table}

\section{Formeln}
\subsection{Pythagoras}
\begin{description}
\item Der Satz des Pythagoras errechnet sich wie folgt: $\ a^2 + b^2 = c^2 $. Daraus können wir die Länge der Hypothenuse wie folgt berechnen: $c=\sqrt{a^2 + b^2}$
\end{description}

\subsection{Summen}
\begin{description}
\item Wir können auch die Formel für eine Summe angeben:
\begin{equation}
s=\sum_{k=1}^n i=(n*(n+1))/2
\end{equation}
\end{description}

\end{document}
